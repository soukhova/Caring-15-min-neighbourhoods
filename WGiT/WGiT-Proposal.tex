% Options for packages loaded elsewhere
\PassOptionsToPackage{unicode}{hyperref}
\PassOptionsToPackage{hyphens}{url}
\PassOptionsToPackage{dvipsnames,svgnames,x11names}{xcolor}
%
\documentclass[
  authoryear,
  preprint,
  3p]{elsarticle}

\usepackage{amsmath,amssymb}
\usepackage{iftex}
\ifPDFTeX
  \usepackage[T1]{fontenc}
  \usepackage[utf8]{inputenc}
  \usepackage{textcomp} % provide euro and other symbols
\else % if luatex or xetex
  \usepackage{unicode-math}
  \defaultfontfeatures{Scale=MatchLowercase}
  \defaultfontfeatures[\rmfamily]{Ligatures=TeX,Scale=1}
\fi
\usepackage{lmodern}
\ifPDFTeX\else  
    % xetex/luatex font selection
\fi
% Use upquote if available, for straight quotes in verbatim environments
\IfFileExists{upquote.sty}{\usepackage{upquote}}{}
\IfFileExists{microtype.sty}{% use microtype if available
  \usepackage[]{microtype}
  \UseMicrotypeSet[protrusion]{basicmath} % disable protrusion for tt fonts
}{}
\makeatletter
\@ifundefined{KOMAClassName}{% if non-KOMA class
  \IfFileExists{parskip.sty}{%
    \usepackage{parskip}
  }{% else
    \setlength{\parindent}{0pt}
    \setlength{\parskip}{6pt plus 2pt minus 1pt}}
}{% if KOMA class
  \KOMAoptions{parskip=half}}
\makeatother
\usepackage{xcolor}
\setlength{\emergencystretch}{3em} % prevent overfull lines
\setcounter{secnumdepth}{5}
% Make \paragraph and \subparagraph free-standing
\ifx\paragraph\undefined\else
  \let\oldparagraph\paragraph
  \renewcommand{\paragraph}[1]{\oldparagraph{#1}\mbox{}}
\fi
\ifx\subparagraph\undefined\else
  \let\oldsubparagraph\subparagraph
  \renewcommand{\subparagraph}[1]{\oldsubparagraph{#1}\mbox{}}
\fi


\providecommand{\tightlist}{%
  \setlength{\itemsep}{0pt}\setlength{\parskip}{0pt}}\usepackage{longtable,booktabs,array}
\usepackage{calc} % for calculating minipage widths
% Correct order of tables after \paragraph or \subparagraph
\usepackage{etoolbox}
\makeatletter
\patchcmd\longtable{\par}{\if@noskipsec\mbox{}\fi\par}{}{}
\makeatother
% Allow footnotes in longtable head/foot
\IfFileExists{footnotehyper.sty}{\usepackage{footnotehyper}}{\usepackage{footnote}}
\makesavenoteenv{longtable}
\usepackage{graphicx}
\makeatletter
\def\maxwidth{\ifdim\Gin@nat@width>\linewidth\linewidth\else\Gin@nat@width\fi}
\def\maxheight{\ifdim\Gin@nat@height>\textheight\textheight\else\Gin@nat@height\fi}
\makeatother
% Scale images if necessary, so that they will not overflow the page
% margins by default, and it is still possible to overwrite the defaults
% using explicit options in \includegraphics[width, height, ...]{}
\setkeys{Gin}{width=\maxwidth,height=\maxheight,keepaspectratio}
% Set default figure placement to htbp
\makeatletter
\def\fps@figure{htbp}
\makeatother

\usepackage{graphicx}
\usepackage{unicode-math}
\usepackage{hyperref}
\def\tightlist{}
\usepackage{setspace}
\doublespacing
\usepackage{lineno}
\linenumbers
\makeatletter
\makeatother
\makeatletter
\makeatother
\makeatletter
\@ifpackageloaded{caption}{}{\usepackage{caption}}
\AtBeginDocument{%
\ifdefined\contentsname
  \renewcommand*\contentsname{Table of contents}
\else
  \newcommand\contentsname{Table of contents}
\fi
\ifdefined\listfigurename
  \renewcommand*\listfigurename{List of Figures}
\else
  \newcommand\listfigurename{List of Figures}
\fi
\ifdefined\listtablename
  \renewcommand*\listtablename{List of Tables}
\else
  \newcommand\listtablename{List of Tables}
\fi
\ifdefined\figurename
  \renewcommand*\figurename{Figure}
\else
  \newcommand\figurename{Figure}
\fi
\ifdefined\tablename
  \renewcommand*\tablename{Table}
\else
  \newcommand\tablename{Table}
\fi
}
\@ifpackageloaded{float}{}{\usepackage{float}}
\floatstyle{ruled}
\@ifundefined{c@chapter}{\newfloat{codelisting}{h}{lop}}{\newfloat{codelisting}{h}{lop}[chapter]}
\floatname{codelisting}{Listing}
\newcommand*\listoflistings{\listof{codelisting}{List of Listings}}
\makeatother
\makeatletter
\@ifpackageloaded{caption}{}{\usepackage{caption}}
\@ifpackageloaded{subcaption}{}{\usepackage{subcaption}}
\makeatother
\makeatletter
\@ifpackageloaded{tcolorbox}{}{\usepackage[skins,breakable]{tcolorbox}}
\makeatother
\makeatletter
\@ifundefined{shadecolor}{\definecolor{shadecolor}{rgb}{.97, .97, .97}}
\makeatother
\makeatletter
\makeatother
\makeatletter
\makeatother
\journal{7th International Conference on Women and Gender in Transportation (WGiT)}
\ifLuaTeX
  \usepackage{selnolig}  % disable illegal ligatures
\fi
\usepackage[]{natbib}
\bibliographystyle{elsarticle-harv}
\IfFileExists{bookmark.sty}{\usepackage{bookmark}}{\usepackage{hyperref}}
\IfFileExists{xurl.sty}{\usepackage{xurl}}{} % add URL line breaks if available
\urlstyle{same} % disable monospaced font for URLs
\hypersetup{
  pdftitle={Towards caring 15-minute neighbourhoods},
  pdfauthor={Anastasia Soukhov; Léa Ravensbergen; Lucía Mejía Dorantes; Antonio Páez},
  pdfkeywords={Accessibility, Mobility of Care, Gender, 15-Minute
City, Care Complete Neighbourhoods},
  colorlinks=true,
  linkcolor={blue},
  filecolor={Maroon},
  citecolor={Blue},
  urlcolor={Blue},
  pdfcreator={LaTeX via pandoc}}

\setlength{\parindent}{6pt}
\begin{document}

\begin{frontmatter}
\title{Towards caring 15-minute neighbourhoods}
\author[1]{Anastasia Soukhov%
\corref{cor1}%
}
 \ead{soukhoa@mcmaster.ca} 
\author[1]{Léa Ravensbergen%
%
}
 \ead{ravensbl@mcmaster.ca} 
\author[2]{Lucía Mejía Dorantes%
%
}
 \ead{lucia.mejia.dorantes@gmail.com} 
\author[1]{Antonio Páez%
%
}
 \ead{paezha@mcmaster.ca} 

\affiliation[1]{organization={McMaster University, School of Earth,
Environment \& Society},city={Hamilton, Canada},postcodesep={}}
\affiliation[2]{organization={Consultant},city={Karlsruhe,
Germany},postcodesep={}}

\cortext[cor1]{Corresponding author}




        





\begin{keyword}
    Accessibility \sep Mobility of Care \sep Gender \sep 15-Minute
City \sep 
    Care Complete Neighbourhoods
\end{keyword}
\end{frontmatter}
    \ifdefined\Shaded\renewenvironment{Shaded}{\begin{tcolorbox}[interior hidden, boxrule=0pt, enhanced, sharp corners, breakable, borderline west={3pt}{0pt}{shadecolor}, frame hidden]}{\end{tcolorbox}}\fi

The 15-Minute City is a normative conceptualisation gaining ground in
urban planning: it frames neighbourhoods as responsive to human needs
and environmental sensibilities, where most daily necessities can be
reached within a 15-minute walk or bike ride \citep{Allam2022}. As a
related tool, accessibility measures (the ease of reaching
opportunities) are increasingly important amongst transport planners
aiming to foster just and sustainable cities
\citep{vale2023accessibility}. Both the 15-Minute City and accessibility
measures are flexible enough to consider all destination types
holistically however, gendered examinations have been lacking in the
literature. For instance, accessibility analyses have historically
focused on employment-centric and discretionary travel, types of travel
more frequent for working-age and higher-income men.

To counter this masculinist bias, this study investigates a way to
gender-mainstream the 15-Minute City through a care lens
\citep[e.g.,][]{lawWomenTransportNew1999, uteng2008gendered, Oestergaard1992, levy1991towards, levy2013travel}
supported by the Mobility of Care framework (coined by
\citet{sanchezdemadariagaMobilityCareIntroducing2013}). Mobility of Care
emphasizes the importance of travel to unpaid work (care trips) in
contrast to the better-studied travel to employment and leisure. While
all three trip types (work, care, and leisure) are essential, care trips
are often relatively shorter-distance, proximate to
residential/work/school, and comprise approximately 30\% of adults'
daily trips
\citep{sanchezdemadariagaMeasuringMobilitiesCare2019, ravensbergen2023exploratory, mejia2021mobility};
fitting well within the 15-Minute City conceptualisation.

Our study provides an empirical example that maps the 15-Minute City
onto the Mobility of Care framework. Specifically, it identifies which
areas in Hamilton, Canada are `caring 15-minute neighbourhoods'. To do
so, a database of care destinations is created using secondary data. In
this database, care destinations include all places associated with
sustaining household tasks needed for the reproduction of life namely:
shopping (e.g., groceries), errands (e.g., libraries), health (e.g.,
dentist), and caring for dependents (e.g., schools). This database is
used to estimate the number and mix of care-destinations that can be
reached within a 15-minute walk- and cycling- sheds from census
centroids. Typologies are generated to illustrate which neighbourhoods
can and to what degree facilitate 15-minute access to care.

Results indicate only a few neighbourhoods outside of the downtown core
are `care-complete', i.e., contain a sufficiently high mix of care
destinations from all categories and sub-categories. However, some
neighbourhoods are almost `care-complete' and provide 15-minute access
to some care categories. Our study frames these neighbourhoods on the
continuum of `caring' and in need of further intervention. The
quantitative investigation conducted provides a high-level picture of
what neighbourhoods (and their underlying land-use) are connected to
transport infrastructure that can support reaching care-destinations.

Taken together, this study provides a theoretical bridge to connect
15-Minute Cities, accessibility analysis and Mobility of Care framework
for the purpose of informing policy choice aimed to encourage just and
sustainable mobility.


\renewcommand\refname{References}
  \bibliography{bibliography.bib}


\end{document}
